\documentclass[11pt]{article}
\usepackage{amssymb}
\usepackage[UTF8]{ctex}
\usepackage{geometry}
\usepackage{units}
\usepackage{pifont}
\geometry{
	a4paper,
	total={150mm,237mm},
	left=30mm,
	top=27mm,
	}
\usepackage{amsmath}
\usepackage{enumerate}
\usepackage{lipsum}
\usepackage{graphicx}
\usepackage{hyperref}
\usepackage{indentfirst}
\usepackage[graphicx]{realboxes}
\usepackage{booktabs}
\usepackage{cases}
\usepackage{subfig}  
\usepackage{float}
\usepackage{pythonhighlight}

\setlength{\parindent}{2em}
\title{HW5}
\author{姓名:陈锐林,学号:21307130148}

\begin{document}
\maketitle
\begin{large}
	\noindent Question1\\
\end{large}
\hspace*{2em}(1)已知基址为1000,空间大小为100,头部大小为0,根据题目的要求,进行一系列的malloc和free后,结果如下表:\\
\begin{tabular}{p{2cm}p{2cm}p{11cm}}  % 其中,tabular是表格内容的环境;c表示centering,即文本格式居中;c的个数代表列的个数
    \toprule[2pt]
    操作 & 返回 & 内存 \\ %中间用 & 隔开, 换行用\\
    \midrule[2pt]
    Alloc(3)    & 1000      & (1003,97)      \\
    free        & 0 & (1000,3),(1003,97) \\
    Alloc(5)    & 1003 & (1000,3),(1008,92) \\
    free        & 0 & (1000,3),(1003,5),(1008,92) \\
    Alloc(8)    & 1008 & (1000,3),(1003,5),(1016,84) \\
    free        & 0 & (1000,3),(1003,5),(1008,8),(1016,84) \\
    Alloc(8)    & 1008 & (1000,3),(1003,5),(1016,84) \\
    free        & 0 & (1000,3),(1003,5),(1008,8),(1016,84) \\
    Alloc(2)    & 1000 & (1002,1),(1003,5),(1008,8),(1016,84) \\
    Alloc(7)    & 1008 & (1002,1),(1003,5),(1015,1),(1016,84) \\
    \bottomrule[2pt]
\end{tabular}
\hspace*{2em}可以注意到最后free list被分成越来越多的小块;并且不可逆。\\
\begin{large}
	\noindent Question2\\
\end{large}
\hspace*{2em}(1)如果换成WORST策略,结果如下表:\\
\begin{tabular}{p{2cm}p{2cm}p{11cm}}  % 其中,tabular是表格内容的环境;c表示centering,即文本格式居中;c的个数代表列的个数
    \toprule[2pt]
    操作 & 返回 & 内存 \\ %中间用 & 隔开, 换行用\\
    \midrule[2pt]
    Alloc(3)    & 1000      & (1003,97)      \\
    free        & 0 & (1000,3),(1003,97) \\
    Alloc(5)    & 1003 & (1000,3),(1008,92) \\
    free        & 0 & (1000,3),(1003,5),(1008,92) \\
    Alloc(8)    & 1008 & (1000,3),(1003,5),(1016,84) \\
    free        & 0 & (1000,3),(1003,5),(1008,8),(1016,84) \\
    Alloc(8)    & 1016 & (1000,3),(1003,5),(1008,8),(1024,76)\\
    free        & 0 & (1000,3),(1003,5),(1008,8),(1016,8),(1024,76) \\
    Alloc(2)    & 1024 & (1000,3),(1003,5),(1008,8),(1016,8),(1026,74) \\
    Alloc(7)    & 1026 & (1000,3),(1003,5),(1008,8),(1016,8),(1033,67)  \\
    \bottomrule[2pt]
\end{tabular}
\hspace*{2em}同样的free list被分成越来越多的小块,且不可逆;且因为WORST策略导致前面的小块不再被使用到,最后剩的块数也多起来了。\\

\begin{large}
	\noindent Question3\\
\end{large}
\hspace*{2em}如果用FIRST策略,因为选择最小的合适块即可,所以会减少搜索次数;这加速了查找。结果如下:\\
\begin{tabular}{p{2cm}p{2cm}p{11cm}}  % 其中,tabular是表格内容的环境;c表示centering,即文本格式居中;c的个数代表列的个数
    \toprule[2pt]
    操作 & 返回 & 内存 \\ %中间用 & 隔开, 换行用\\
    \midrule[2pt]
    Alloc(3)    & 1000      & (1003,97)      \\
    free        & 0 & (1000,3),(1003,97) \\
    Alloc(5)    & 1003 & (1000,3),(1008,92) \\
    free        & 0 & (1000,3),(1003,5),(1008,92) \\
    Alloc(8)    & 1008 & (1000,3),(1003,5),(1016,84) \\
    free        & 0 & (1000,3),(1003,5),(1008,8),(1016,84) \\
    Alloc(8)    & 1008 & (1000,3),(1003,5),(1016,84) \\
    free        & 0 & (1000,3),(1003,5),(1008,8),(1016,84) \\
    Alloc(2)    & 1000 & (1002,1),(1003,5),(1008,8),(1016,84) \\
    Alloc(7)    & 1008 & (1002,1),(1003,5),(1015,1),(1016,84) \\
    \bottomrule[2pt]
\end{tabular}\\

\begin{large}
	\noindent Question4\\
\end{large}
\hspace*{2em}(1)FIRST策略是选择找到的第一个足够大的空间,所以很难去判断会怎样受到排序策略的影响。能确定的是SIZESORT-下是最快的,因为第一块应是最能满足的;其余排序则要考虑所需块的大小。(2)BEST策略在SIZESORT+下表现最好,因为这样遍历过去,第一个满足空间大小的就是所求。但是因为BEST是找满足大小的最小块,所以实际需要的时间会因为具体的free list和所需大小而浮动。只是SIZESORT+是最顺的。(3)WORST在SIZESORT-下最好,第一个一定就是满足要求的(或者再也找不到了)。(4)如果要考虑合并的话ADDSORT可能会是一个好策略。\\

\begin{large}
	\noindent Question5\\
\end{large}
\hspace*{2em}(1)较大的分配需求可能会失败,因为不合并的话最后free list是支离破碎的小块。(2)在有-C标志的情况下,相邻的空闲块得到合并;能应付(不合并时无法应付的)较大的分配需求。
(3)沿用前面的命令,只修改-n(FIRST策略)。在允许合并时,每行的free list都很简洁;最后剩下一块90的大小;不允许合并时,中间出现了分配失败的情况;最后free list冗长且每块很小,最后大小是不到90的。
(4)这时,列表的顺序只有当采用FIRST策略时才会对free list有影响。其他策略不受影响。\\

\begin{large}
	\noindent Question6\\
\end{large}
\hspace*{2em}当改变 -P 时,高于50,说明分配操作要比释放多;接近100,意味着几乎所有操作都是分配操作;接近0时,仍近乎是一半分配一半释放,因为只有分配了才能释放。\\

\begin{large}
	\noindent Question7\\
\end{large}
\hspace*{2em}(1)指令:python3 ./malloc.py -c -A +10,+15,+20,+25,+30,-0,-1,-2,-3,-4; 最后free list为:[ addr:1000 sz:10 ][ addr:1010 sz:15 ][ addr:1025 sz:20 ][ addr:1045 sz:25 ][ addr:1070 sz:30 ]。
\par(2)指令:python3 ./malloc.py -c -A +10,+15,+20,+25,+30,-0,-1,-2,-3,-4 -l SIZESORT+ -C; 最后free list为:[ addr:1025 sz:20 ][ addr:1000 sz:25 ][ addr:1045 sz:55 ]。
\par(3)指令:python3 ./malloc.py -c -A +10,+15,+20,+25,+30,-0,-1,-2,-3,-4 -l SIZESORT- -C; 最后free list为:[ addr:1070 sz:30 ][ addr:1045 sz:25 ][ addr:1025 sz:20 ][ addr:1010 sz:15 ][ addr:1000 sz:10 ]。\\
\end{document}